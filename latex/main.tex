% !TeX encoding = UTF-8
% !TeX program = pdflatex
% !TeX spellcheck = it_IT

% \documentclass[a4paper, 11pt]{article}
\documentclass[Lau,binding=0.6cm]{sapthesis}

\usepackage{microtype}
\usepackage[italian]{babel}
\usepackage[utf8]{inputenx}

\usepackage{hyperref}
\hypersetup{pdftitle={tesi},pdfauthor={Andrei Laurentiu Lepadat}}

\title{Using Sensor and Process Noise Fingerprint to Detect Cyber Attacks in CPS}
\author{Andrei Laurentiu Lepadat}
\IDnumber{1677093}
\course{Informatica}
\courseorganizer{Facoltà di Ingegneria dell'Informazione, Informatica e Statistica}
\AcademicYear{2020/2021}
\copyyear{2021}
\advisor{Prof. Enrico Tronci}
\authoremail{lepadat.1677093@studenti.uniroma1.it}

\versiondate{\today}

\begin{document}

\frontmatter

\maketitle

\dedication{Decidere se inserire. Ne vale la pena?}

% ----------------------------------------------------------------------------------------

\tableofcontents

\chapter{Sommario} 
% Massimo una pagina da scrivere alla fine

% ----------------------------------------------------------------------------------------   

\mainmatter

\chapter{Introduzione} 
% Scrivere alla fine 

\section{Contesto}
% Short description of context,  for example:
% Tutti gli essere vivneti per sopravviver hanno bisogno di alimentarsi.
% Gli uomini sono essere viventi, e quindi hanno bisogno di alimentarsi.

\section{Motivazioni}
% Motivation for what we want to do. esempio:
% Purtroppo la mancanza di una alimentazione adeguata è uno dei problemi più importanti nei paesi sottosviluppati

\section{Contributi}
% Describe here your contributions, namely the missing items identified in the motivations. esempio:
% Questo lavoro propone un metodo per generare cibo per tutti gli abitanti dei paesi sottosviluppati ...

\section{Lavori correlati}
% Describe here the state of the art (e.g., algorithms and tools  available).
% For each paper/tool explain what you do that is not already available (killing).

\section{Struttura}
% Give an outline of the thesis structure (one sentence per section)


% ----------------------------------------------------------------------------------------

\chapter{Background}
% Put in this section all the background knowledge needed to understand what you did.

% Parlare di dati di sensori, serie temporali estratti dai sensori, rmse, rumore intrinseco sensori --> residual
% machine learning model e decision 

Ogni sistema cyber-fisico che si rispetti è dotato di almeno un sensore che ha il compito di misurare una determinata ``qualità'' fisica di interesse per il sistema stesso. 
I dati che vengono rilevati dai sensori spesso vengono memorizzati localmente e/o in modo remoto e possono essere impiegati,
come nel lavoro qui presentato, per fini paralleli o trasversali a quelli per cui sono stati installati.
Una sequenza di dati estratti da sensori ordinata temporalemente viene chiamata \textit{serie temporale} (\textit{time-series} in inglese).

Comunemente i sensori sono imperfetti per costruzione e trasportano intrinsecamente un'incertezza (rumore) che influenza le misurazioni da essi compiute.
Sia 
$$
\bar{y}_{k} = y_{k} + \delta_{k}
$$
il valore misurato da un determinato sensore nell'istante di tempo \textit{k}, composto da $y_k$, il valore effettivo in quell'istante della grandezza misurata, più $\delta_k$, il rumore aggiunto.

In un determinato istante di tempo, il valore di ogni sensore del sistema costituisce lo \textit{stato} del sistema.
La sfida di estrarre il fingerprint dai sensori è data dal fatto che questi stati sono dinamici. 
Prendendo in considerazione, per esempio, un termometro, se la temperatura dell'ambiente che misura rimane costante nel tempo è facile estrarre

% ----------------------------------------------------------------------------------------

\chapter{Metodi}
% Describe here the algorithm design from a math perspective. No code here.

% ----------------------------------------------------------------------------------------

\chapter{Implementazione}
% describe here how you implemented the functionalities described in the methods section.
% Use pseudo-code or diagrams.

% Parlare del come si è implementato il rumore intrinseco in un modello symulink

% ----------------------------------------------------------------------------------------

\chapter{Risultati sperimentali}
% describe the experimental results

\section{Obiettivi}
% describe here the objectives of the experimental activity: test correctness;
% evaluate computational performances (CPU time and RAM); ....

\section{Configurazione (Setting) (?)}
% describe experimental setting: hardware used, software used, data used, etc. ....

\section{Casi di studio}
% describe here the case studies used in the experiments

\section{Correttezza}
% describe here experiments aimed at showing correctness of your implementation
% Risultati cross-validation

\section{Valutazione computazionale}
% describe here experiments aimed at evaluating computational performance (e.g., CPU usage, RMA usage, etc)

\section{Valutazione tecnica}
% describe here experiments aimed at evaluating your tools wrt their intended use.
% For example, software to detect adversarial attacks will be evaluated here showing how good (or bad) it is in detecting attcks.

\chapter{Conclusioni}
% Sum up what you did and outline some possible future developments.

\backmatter
% Inserire qui referenze e citazioni

\end{document}