% !TeX encoding = UTF-8
% !TeX program = pdflatex
% !TeX spellcheck = it_IT

% \documentclass[a4paper, 11pt]{article}
\documentclass[Lau,binding=0.6cm]{sapthesis}

\usepackage{microtype}
\usepackage[italian]{babel}
\usepackage[utf8]{inputenx}
\usepackage[hidelinks]{hyperref}
\usepackage{amssymb}
\usepackage{hyperref}
\hypersetup{pdftitle={tesi},pdfauthor={Andrei Laurentiu Lepadat}}

\title{Using Sensor and Process Noise Fingerprint to Detect Cyber Attacks in CPS}
\author{Andrei Laurentiu Lepadat}
\IDnumber{1677093}
\course{Informatica}
\courseorganizer{Facoltà di Ingegneria dell'Informazione, Informatica e Statistica}
\AcademicYear{2020/2021}
\copyyear{2021}
\advisor{Prof. Enrico Tronci}
\authoremail{lepadat.1677093@studenti.uniroma1.it}

\versiondate{\today}

\begin{document}

\frontmatter

\maketitle

\dedication{Decidere se inserire. Ne vale la pena?}

% ----------------------------------------------------------------------------------------

\tableofcontents

\chapter{Sommario} 
% Massimo una pagina da scrivere alla fine

% ----------------------------------------------------------------------------------------   

\mainmatter

\chapter{Introduzione} 
% Scrivere alla fine 

\section{Contesto}
% Short description of context,  for example:
% Tutti gli essere vivneti per sopravviver hanno bisogno di alimentarsi.
% Gli uomini sono essere viventi, e quindi hanno bisogno di alimentarsi.

\section{Motivazioni}
% Motivation for what we want to do. esempio:
% Purtroppo la mancanza di una alimentazione adeguata è uno dei problemi più importanti nei paesi sottosviluppati

\section{Contributi}
% Describe here your contributions, namely the missing items identified in the motivations. esempio:
% Questo lavoro propone un metodo per generare cibo per tutti gli abitanti dei paesi sottosviluppati ...

\section{Lavori correlati}
% Describe here the state of the art (e.g., algorithms and tools  available).
% For each paper/tool explain what you do that is not already available (killing).

\section{Struttura}
% Give an outline of the thesis structure (one sentence per section)


% ----------------------------------------------------------------------------------------

\chapter{Background}
% Put in this section all the background knowledge needed to understand what you did.

% Parlare di dati di sensori, serie temporali estratti dai sensori, rmse, rumore intrinseco sensori --> residual
% machine learning model e decision 

Ogni sistema cyber-fisico che si rispetti è dotato di almeno un sensore che ha il compito di misurare una determinata ``qualità'' fisica di interesse per il sistema stesso. 
I dati che vengono rilevati dai sensori spesso vengono memorizzati localmente e/o in modo remoto e possono essere impiegati,
come nel lavoro qui presentato, per fini paralleli o trasversali a quelli per cui sono stati installati.
Una sequenza di dati estratti da sensori ordinata temporalmente viene chiamata \textit{serie temporale} (\textit{time-series} in inglese).

Comunemente i sensori sono imperfetti per costruzione e trasportano intrinsecamente un'incertezza (rumore) che influenza le misurazioni da essi compiute.
Sia 
\begin{equation}
\bar{y}_{k} = y_{k} + \delta_{k}\label{eq:1}
\end{equation}
il valore misurato da un determinato sensore nell'istante di tempo \textit{k}, composto da $y_k$, il valore effettivo in quell'istante della grandezza misurata, più $\delta_k$, il rumore aggiunto.

In un determinato istante di tempo, il valore di ogni sensore del sistema costituisce lo \textit{stato} del sistema.
La sfida di estrarre il fingerprint dai sensori è data dal fatto che questi stati sono dinamici. 
Prendendo in considerazione, per esempio, un termometro, se la temperatura dell'ambiente che misura rimane costante nel tempo è facile estrarre il fingerprint del rumore e costruirne il profilo, 
ma in processi reali non è così semplice, gli stati cambiano continuamente, per esempio l'aumento di velocità di una macchina per via della pressione sul pedale dell'acceleratore.
\`E importante catturare queste variazioni affinch\'e le misurazioni dinamiche dei sensori possano essere stimate.
In [1] questo problema viene affrontato definendo un modello analitico del sistema interessato, rappresentato tramite il modello \textit{State-Space}. 
Vengono implementate le tecniche definite in [2], definendo cos\`i il modello lineare tempo inviariante (LTI) del sistema, rappresentato dal sistema di equazioni
\begin{equation}
    \begin{cases}
        x_{k+1} = Ax_k + Bu_k + \vartheta_k, \\
        y_k = Cx_k + \eta_k:
    \end{cases}
\end{equation}
in cui $x_k \in \mathbb{R}^n$ rappresenta lo stato del sistema, $u_k \in \mathbb{R}^p$ l'input di controllo e
$\vartheta_k$ il rumore al tempo $k$.
$y_k \in \mathbb{R}_m$ e $\eta_k \in \mathbb{R}_m$ rappresentano, rispettivamente, la misurazione e il rumore del sensore al tempo $k$.
$A$, $B$, $C$ sono le matrici dello spazio di stato di dimensioni adeguate che rappresentato la dinamica del sistema.

Definito il precedente sistema, ci sono molti punti che un attaccante mal intenzionato potrebbe bersagliare.
Nel lavoro presentato, cos\`i come in [1], vengono presi in considerazione \textit{spoofing attack} ai sensori che potrebbero essere portati a termine tramite uno schema \textit{Man-in-The-Middle}.
L'equazione lineare che rappresenta questa tipologia di attacchi \`e data da

\begin{equation}
\bar{y}_{k} = y_{k} + \delta_{k} = Cx_k + \eta_k + \delta_k,\footnote{Notare l'ugualianza con l'equazione \ref{eq:1}: un attacco \`e considerato come un'introduzione di rumore nella misurazione fatta da un sensore.}
\end{equation}
in cui $\delta_k \in \mathbb{R}_m$ rappresenta un attacco ai sensori.

In [1], dato l'output $\bar{y}_k$, viene adoperato il \textit{filtro di Kalman} per stimare lo stato del sistema e il vettore dei \textit{residui}, definito, in questo contesto, come la differenza tra la reale misurazine effettuata dal sensore
e la stima della misurazione calcolata dal filtro nell'istante $k$:
\begin{equation}
    r_k := \bar{y}_k - \hat{y}_k,
\end{equation}
dove $\hat{y}_k$ \`e l'output del filtro di Kalman.

Detto ci\`o, per quantificare la bont\`a del modello del sistema, viene utilizzato l'\textit{Errore Quadratico Medio (RMSE)}, definito come
\begin{equation}
    RMSE = \sqrt{\frac{\sum_{i=1}^n (y_i - \hat{y}_i)}{n}}.
\end{equation}
Questa metrica rappresenta la distanza tra il valore stimato e quello misurato, ovvero quanto il primo \`e lontano dal secondo.
Nella letteratura della teoria del controllo, modelli con un'accuratezza superiore al 70\% sono considerati accettabili approssimazioni della dinamica di sistemi reali.

Per ogni momento statistico (media, deviazione standard, \ldots) di una serie storica (ma non solo) si pu\`o definire un \textit{intervallo di confidenza} 
che esprime la probabilit\`a che il valore calcolato sugli $N$ campioni della serie approssimi il valore effettivo del momento statistico.
Questo intevallo dipende, nel caso del valore medio, si definisce come 
\begin{equation}
    Pr\{\bar{x} - \epsilon \leq \mu \leq \bar{x} + \epsilon\} = 1 - \delta,
    \label{eq:2}
\end{equation}
in cui $\mu$ e $\bar{x}$ sono, rispettivamente, la media effettiva e quella calcolata. $\epsilon$ e $\delta$ sono valori che dipendono da $N$, e mantenendo $\delta$ costante
e incrementando $N$, anche $\epsilon$ cresce, allargando l'intervallo di confidenza.
Tale intervallo di condifenza pu\`o essere definito anche per momenti di ordine superiore.

Nel contesto del presente lavoro, come si vedr\`a, volendo giudicare la legittimit\`a delle misurazioni di un determinato sensore, determinate propriet\`a statistiche delle nuove misurazioni (nuove nel contesto di normale 
funzionamento del sistema aperto ad attacchi) verranno confrontate con le stesse propriet\`a di misurazioni effettuate in condizioni \textit{sicure} (questi valori sono chiamati valori di \textit{reference}). 
Per le nuove misurazioni, prendendo ancora in esempio il valore medio e volendo avere un intervallo di confidenza il pi\`u piccolo possibile (quindi un $\epsilon$ il pi\`u piccolo possibile), bisogna essere attenti per via di valori di $N$ non molto grandi,
caratteristica preferibile in quanto non si vogliiono campionare troppi valori in situazioni real-time.
 
\ldots bisogner\`a affrontare il problema avvalendosi dell'aiuto di un determintao modello di \textit{Machine Learning}, che d\`a buone approssimazioni

Il vettore dei residui \`e quindi parte fondamentale per la definizione dei fingerprint dei sensori.
A tale scopo viene definito un problema di M.L. che ha come \textit{feature} alcuni valori statistici estratti dai vettori residui.
Queste feature sono mostrate nella Tabella \ref{tab:1}.

\begin{table}[tb]
    \begin{center}
    \begin{tabular}{|l|l|}
    \hline
    Feature & Descrizione \\
    \hline
    Media & $\bar{x} = \frac{1}{N}\sum_{i=1}^N x_i$ \\
    \hline
    Varianza & $\sigma = \frac{1}{N}\sum_{i=1}^N (x_i - \bar{x})^2 $ \\
    \hline
    Dev. Med. Ass. & $D_{\bar{x}} = \frac{1}{N}\sum_{i=1}^N |x_i - \bar{x}|$ \\
    \hline
    Asimmetria & $\gamma = \frac{1}{N} \sum_{i=1}^N (\frac{x_i - \bar{x}}{\sigma})^3 $ \\
    \hline
    Curtosi & $ \beta = \frac{1}{N} \sum_{i=1}^N (\frac{x_i - \bar{x}}{\sigma})^4 - 3$\\
    \hline
    \end{tabular}
    \end{center}
    \caption{Lista delle feature utilizzate; $x$ \`e la serie temporale di dimensione $N$ proveniente dal sensore.}
    \label{tab:1}
\end{table}

Un problema di M.L. pu\`o essere definito come una funzione $f: X \to Y$, dato un insieme $D$ (dataset) contenente informazioni riguardanti $f$.
Fare il \textit{learning} della funzione $f$ significa trovare un'altra funzione $\hat{f}$ che approssima e ritorna valori più vicini possibile ad $f$, specialmente per elementi non presenti in $D$.
Nel presente lavoro il problema viene definito come un problema \textit{supervised}\footnote{$D = \{ (x_i, y_i)_{i=1}^N \}$} di \textit{classificazione}, cio\`e in cui $f$ \`e definita tale che
\begin{equation}
    \begin{array}{l}
    X := \mathbb{R}^m, \\
    Y := \{ C_1, C_2, \ldots, C_k \}.
    \end{array}
\end{equation}
Quindi $f$ associa ad ogni elemento di $X$, cio\`e un vettore di $m$ reali, un elemento di $Y$, cio\`e la classe $C_i$ di appartenenza.

% Noi calcoliamo i valori statistici derivanti dal sistema quando il sistema non è sotto attacco (la reference) su N campioni. 
% Quando si vanno a calcoalre le features avremo una certa confidenza statistica (un certo epsilon di errore) 
% con confidenza statistica almeno 1-delta la media (per esempio) è il valore trovato +- epsilon, epsilon e delta dipendono da N.
% Pr{\bar{x} - \epsilon <= \mu <= \bar{x} + \epsilon} = 1 - \delta
% Questi numeri rappresentano un intervallino. (alta confidenza statistica -> piccolo intervallino)
% Quando facciamo le operazioni normali, non vogliamo prendere un epsilon troppo grande, perché ci vorrebbe tropppo tempo per rilevare un attacco 
% L'intervallo calcolato durante il normale funzionamento potrebbe avere un'intersezione non vuota con quello della reference (no attacco)
% A questo punto non sappiamo se c'è attacco o no, qui entra in gioco SVM perché...

% Nel presente lavoro, vengono definiti due problemi di M.L., uno \textit{Supervisionato} e uno \textit{Non Supervisionato}.
% Nel primo caso il dataset \`e composto da coppie del tipo $D = \{ (x_i, y_i)_{i=1}^N \}$, nel secondo caso del tipo $D = \{ (x_i)_{i=1}^N \}$.

% ----------------------------------------------------------------------------------------

\chapter{Metodi}
% Describe here the algorithm design from a math perspective. No code here.

% ----------------------------------------------------------------------------------------

\chapter{Implementazione}
% describe here how you implemented the functionalities described in the methods section.
% Use pseudo-code or diagrams.

% Parlare del come si è implementato il rumore intrinseco in un modello symulink
% Parlare di come è stato suddiviso in chunks il vettore dei residui
% ----------------------------------------------------------------------------------------

\chapter{Risultati sperimentali}
% describe the experimental results

\section{Obiettivi}
% describe here the objectives of the experimental activity: test correctness;
% evaluate computational performances (CPU time and RAM); ....

\section{Configurazione (Setting) (?)}
% describe experimental setting: hardware used, software used, data used, etc. ....

\section{Casi di studio}
% describe here the case studies used in the experiments

\section{Correttezza}
% describe here experiments aimed at showing correctness of your implementation
% Risultati cross-validation

\section{Valutazione computazionale}
% describe here experiments aimed at evaluating computational performance (e.g., CPU usage, RMA usage, etc)

\section{Valutazione tecnica}
% describe here experiments aimed at evaluating your tools wrt their intended use.
% For example, software to detect adversarial attacks will be evaluated here showing how good (or bad) it is in detecting attcks.
% Parlare della grandezza dei chunks trovati

\chapter{Conclusioni}
% Sum up what you did and outline some possible future developments.

\backmatter
% Inserire qui referenze e citazioni

\end{document}